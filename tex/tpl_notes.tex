\title{Advanced Topics in Programming Language - Duck Typing in Python}
\author{
        Tommaso Puccetti \\
                Studente presso Universita degli studi di Firenze
}
\date{\today}
\documentclass[12pt]{article}
\usepackage{graphicx}
\usepackage{hyperref}
\usepackage{listings}
\usepackage{jlcode}

\begin{document}
\maketitle
\tableofcontents
\listoftables
\listoffigures

\section{Possible references}
	\begin{itemize}
		\item \href{https://www.youtube.com/watch?v=fK5lcaNqdj4}{You Tube video};
		\item \href{https://hackernoon.com/python-duck-typing-or-automatic-interfaces-73988ec9037f}{Python duck typing (or automatic interfaces)- how change the dependecy injection};
		\item \href{https://medium.com/programming-hacks/duck-typing-in-python-6740aa72b301}{Simple example};
		\item \href{http://www.voidspace.org.uk/python/articles/duck_typing.shtml}{Something more technical};
		\item 
		\href{https://realpython.com/python-type-checking/}{Ultimate guide to Python's type checking;}
	\end{itemize}
	
\section{Type checking: recall }
	
		\textbf{Type checking} is the process of verifying and enforces the typing rules of a language. In other words the \textbf{type checker} (the type checking algorithm of the language ) is used to prove the \textbf{type safety} of a program.\\
		It may occur either at:
		
		\begin{enumerate}
			\item compile time (\textbf{Static});
			\item run time (\textbf{Dynamic}). 
		\end{enumerate}
	
		Let's see in details what's the difference:	
		
		\begin{itemize}
			\item \textbf{Static type checking: }is the process of verifying the type safety of a program based on the analysis of a program text.  If a program passes a static type checker, then the program is guaranteed to satisfy some set of type safety properties for all possible inputs.
			\item \textbf{Dynamic type checking:} is the process of verifying the type safety of a program at runtime. It may cause a program to fail at runtime.
		\end{itemize}
	
		There are also two different ways to classify the type check:
		
		\begin{itemize}
			\item \textbf{Explicitly typed:} each variables is annotaded in source code with type's information. In this case the \textit{type check is simple but the language is more difficult} (from the programmers point of view).
			\item \textbf{Implicitly typed:} the data types of source code are automatically detected. It is also rederred as \textbf{type inference}. The language \textit{is easier but the type check algorithm is far more complex}.
		\end{itemize}
	
		\subsection{Example: type inference vs. dynamic typing}
			These two kind of typings could be confused. Here an example to clarify the differences:

			\begin{lstlisting}
				var1 = 10
				var2 = "astring"
				var3 = var1 + var2
			\end{lstlisting}
			
			\begin{enumerate}
				\item In \textbf{dynamically typed} language this code run without errors: at runtime the \textit{var1} is forced to be a string and the result is \textit{"10astring"};
				\item By the other side, in \textbf{inferred type language} the compiler \textit{throw an error}.
			\end{enumerate}
		







		
\end{document}